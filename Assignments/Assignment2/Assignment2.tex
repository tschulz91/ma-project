\documentclass[11pt]{article}

\usepackage{fullpage}
\usepackage{hyperref}

\title{Econ 899: Assignment 2}
\author{Tim Schulz}
\date{May 27, 2016}

\begin{document}
\maketitle

Between 2007 and 2015, the number of smartphones sold globally increased by a factor of more than 10.\footnote{\url{http://www.statista.com/statistics/263437/global-smartphone-sales-to-end-users-since-2007/}}  But a person can only use so many smartphones at once. For the first time in years, sales growth is expected to fall below double digits.\footnote{\url{https://www.idc.com/getdoc.jsp?containerId=prUS41061616}} This suggests that merely measuring growth through sales will soon be difficult and only provide limited insight.

Alternatively, growth could be measured by usage. It is impractical (if not impossible) to keep track of everybody's smartphone usage. However, by using Wikipedia as a globally representative website (at least for countries where it is not banned), this can be done. Since the end of 2007, the Wikimedia Foundation has been publishing hourly page traffic for all its websites including information on whether an article was accessed in its traditional format or as a mobile site and what language the article is published in.\footnote{Source: \url{https://dumps.wikimedia.org/other/pagecounts-raw/}}

I will use this data to describe the share of the amount of data transmitted due to and the number of mobile visits relative to traditional web traffic. Given the number of observations available, it is possible to estimate this share non-parametrically and to describe growth as its (instantaneous) change. Additionally, I will (attempt to) show how the transition from traditional devices to mobile devices in people's internet usage depends on the time of day\footnote{This only works for languages that are highly concentrated within one or two time zones.} and on the day of the week.

For the higher frequency data, the main explanatory variable will be time itself while also controlling for the increase in the total number of articles (possibly per language). For the purpose of a data description, page traffic aggregated to quarterly or annual data can be investigated as a function of smartphone sales or properties of a country's population.\footnote{•} only works for countries whom a language can uniquely be attributed to.}

\end{document}