\documentclass[11pt]{article}

\usepackage{fullpage}
\bibliographystyle{alpha}


\title{Econ 899: Assignment 3 -- Literature Review}
\author{Tim Schulz}
\date{June 17, 2016}



\begin{document}
\maketitle

\textbf{Context:} I am going to analyse technological diffusion in the context of mobile internet traffic. The newly introduced technologies are different major devices (e.g. first iPhone, first mainstream Android phone) and iterations of mobile operating systems. Since purchasing new devices does not automatically mean that people use it to the fullest extent, technological adoption is measured in terms of usage. The proxy for usage will be mobile internet traffic to and from Wikipedia.
\\
------
\bigskip

Unsurprisingly, processes revolving around technology have been the subject of interest in economic research for some time. However, in most cases ``technology" is considered in the context of production usually as a factor in the production function that increases productivity. Technology in the more colloquial, every-day sense of the word is not the same. From here on, ``technology" will refer to this use of the word.

Innovation, the introduction of new technology, is not the immediate subject of this paper. Rather, I will focus on the process that leads from innovation to widespread adoption of technology: technology diffusion. This process was first quantified by Griliches (1957)\cite{Griliches1957} using the cumulative logistic distribution function in the description of the adoption of hybrid corn in the US. This method not only matches the often observed S-curve of technological adoption over time,\footnote{Initial incorporation of the new technology is slow, then speeds up and eventually slows down again as the ``market" (whatever the context) gets close to being saturated.} but also offers the mathematical convenience of being able to identify a unique ``rate of acceptance".

This concept of technology diffusion has since been applied to areas such as computers (Caselli and Coleman (2001)\cite{CaselliColeman2001}) and hard drives (Christensen (1992)\cite{Christensen1992}). While the latter of these two papers is specifically written in the context of firms where new technology results in a higher number of bits written on a given area of a hard drive, it can be easily translated into the context of consumer electronics: instead of a measure of hard drive performance, the dependent variable in this paper will be mobile internet traffic. Just as the technological concept of a hard drive had already existed for a long time but the ``output" (being bit density) kept improving, so has the concept of mobile internet access existed for some years but the adoption of this technology is still in progress.

Perhaps more related work has been done by Majumdar and Venkataraman (1998)\cite{MajumdarVenkataraman1998}. They analyse technological adoption and diffusion in the context of the telecommunications industry. In this case, technology diffusion is not just a functions of time and knowledge (and other variables) but is specifically subject to network effects. Hall and Khan (2002)\cite{HallKhan2002} extend this idea by introducing the concept of indirect network effects using the example of DVD players. Even if the technology exists and people are, in principle, prepared to adopt it, only once enough people are interested in buying DVDs, it is worthwhile for production companies to switch over from video cassettes. In turn, only then might some people be willing to invest in a DVD player. The same can be argued to be true for mobile internet. While Wikipedia articles did not become available in mobile formats one by one\footnote{The website code only has to be adjusted once to redirect visitors with mobile devices to a differently formatted version of the article (regardless of the article itself).}, there are still network effects associated with other websites. If a user knows from past experiences that most websites are poorly formatted for mobile devices, he is less likely to access the internet from a smartphone even if the the website in question has a mobile version. If, however, (almost) all major websites have mobile versions, a user is much more likely to access the internet using a mobile device than was the case before. Mobile internet usage can therefore be argued to be a function of something best described as the ``overall mobile internet experience". Due to these network effects, Wikipedia page traffic can be seen not only as representative for the rest of the internet but also as a function of mobile services provided my other website operators.


\begin{thebibliography}{99}
\bibitem{Griliches1957}
		Griliches, Zvi, 1957, ``Hybrid Corn: An Exploration in the Economics of Technological Change",
		\textit{Econometrica}, Vol. 25-4, pp. 501--522.
		
\bibitem{CaselliColeman2001}
		Caselli, Francesco, and Wilbur Coleman II, 2001, ``Corss-country Technology Deffusion: The Case of Computers", \textit{American Economic Review} 92-2, pp. 328--335.

\bibitem{Christensen1992}
		Christensen, Clayton M., 1992, ``Exploring the Limits of the Technology S-Curve. Part I: Component Technologies", \textit{Product and Operations Management} 1-4, pp. 334--357.
		
\bibitem{MajumdarVenkataraman1998}
		Majumdar, Sumit, and Venkataraman S., 1992, ``Network Effects and the Adoption of New Technology: Evidence from the U.S. Telecommunications Industry",
		\textit{Strategic Management Journal} 19, pp. 1045--1062.
		
\bibitem{HallKhan2002}
		Hall, Bronwyn, and Khan, Beethika, 2002, ``Adoption of New Technology", \textit{New Economy Handbook}. %%% Find proper reference%%%%%%%%%%%%
		
\end{thebibliography}



\end{document}